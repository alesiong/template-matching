%!TEX TS-program = xelatex
%!TEX encoding = UTF-8 Unicode

\documentclass[12pt, a4paper]{article}

\usepackage{fontspec, xltxtra, xunicode}
\usepackage{amsfonts}
\usepackage{amssymb}
\usepackage{amsmath}
\usepackage{geometry}
\usepackage{clrscode3e}
% \geometry{left=1.25in, right=1.25in, top=1in, bottom=1in}

\usepackage{enumerate} % For listing things
\newfontfamily{\C}{STSongti-SC-Regular}
% \setlength\parindent{0pt}
\usepackage{listings}
\lstset{
  basicstyle=\footnotesize\ttfamily
}
\author{
  \begin{tabular}{c c c}
    QIU Yuheng {\C 邱渝桁} & YU Pengfei {\C 余鹏飞} & YU Xianggang {\C 余湘港} \\
    115010216              & 115010271              & 115010273
  \end{tabular}
}
\title {Course Project Report: Fast Template Matching}
\date{}
\begin{document}
\maketitle
\section{Problem Description}
\section{Algorithm}
  \subsection{Supporting non-square templates}
    The only difference between square templates and non-square templates is how
    to calculate the feature vector. The formula of mean for the square templates
    is:
    \[v_1(D)=\frac{1}{K^2}S_1(D)\]
    The $K^2$ here means the area of $D$, so the formula of mean for the non-square
    template with size $K_x\times K_y$ is:
    \[v_1(D)=\frac{1}{K_xK_y}S_1(D)\]
    Similarly
    \[v_2(D)=\frac{1}{K_xK_y}S_2(D)-v_1(D)^2\]
    Things are little tricky for the gradients. From the original definition,
    the Extended Prewitt Operator on $x$ for the $K_x\times K_y$ patch is:
    \[EP_x(x, y)=x-\frac{K_x+1}{2}\]
    The absolute sum of $EP_x$ is:
    \[\sum_{i=1}^{K_x}\sum_{i=1}^{K_y}\left|EP_x(i,j)\right|\]
    \[=K_y\sum_{i=1}^{K_x}\left|EP_x(i,j)\right|\]
    \[=K_y\frac{(K_x+1)(K_x-1)}{4}\]
    So the gradient on $x$ is:
    \[G_x(D)=\sum_{i=1}^{K_x}\sum_{i=1}^{K_y}EP_x(i,j)I(i,j)/
      \sum_{i=1}^{K_x}\sum_{i=1}^{K_y}\left|EP_x(i,j)\right|\]
    \[=\sum_{i=1}^{K_x}\sum_{i=1}^{K_y}I(i,j)(x-\bar{x})/
      \sum_{i=1}^{K_x}\sum_{i=1}^{K_y}\left|EP_x(i,j)\right|\]
    where $x$ is the global index w.r.t. corresponding local index $i$ within the
    patch and $\bar{x}$ is the median of global index $x$ in this patch
    \[=\frac{4}{K_y(K_x+1)(K_x-1)}(S_x(D)-\bar{x}S_1(D))\]
    So similar as the formula (9) in the Project Paper, $v_3$ and $v_4$ can be
    written as:
    \[v_3(D)=\frac{4}{K_yK_x^2}(S_x(D)-xS_1(D))\]
    \[v_4(D)=\frac{4}{K_xK_y^2}(S_y(D)-yS_1(D))\]
\section{Analysis}
\section{Experiments}
  \subsection{Data}
  \subsection{Findings}


\end{document}
